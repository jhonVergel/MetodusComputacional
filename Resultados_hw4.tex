\documentclass{article}
\usepackage[utf8]{inputenc}

\title{gaficas}
\author{Jhon}
\date{November 2017}

\usepackage{natbib}
\usepackage{graphicx}

\begin{document}

\maketitle

\section{Graficas en orden }


\begin{figure}[hf!]
\centering
\includegraphics[scale=0.5]{./fijas/0.jpg}
\caption{cuerda fija en 0 }
\label{fig:1}
\end{figure}

\begin{figure}[hf!]
\centering
\includegraphics[scale=0.5]{./fijas/1.jpg}
\caption{cuerda fija en T/8 }
\label{fig:2}
\end{figure}

\begin{figure}[hf!]
\centering
\includegraphics[scale=0.5]{./fijas/2.jpg}
\caption{cuerda fija en T/4 }
\label{fig:3}
\end{figure}

\begin{figure}[hf!]
\centering
\includegraphics[scale=0.5]{./fijas/3.jpg}
\caption{cuerda fija en T/2 }
\label{fig:4}
\end{figure}

\begin{figure}[hf!]
\centering
\includegraphics[scale=0.5]{./perturbada/4.jpg}
\caption{cuerda perturbada en 0 }
\label{fig:5}
\end{figure}

\begin{figure}[hf!]
\centering
\includegraphics[scale=0.5]{./perturbada/5.jpg}
\caption{cuerda perturbada en T/8 }
\label{fig:6}
\end{figure}

\begin{figure}[hf!]
\centering
\includegraphics[scale=0.5]{./perturbada/6.jpg}
\caption{cuerda perturbada en T/4 }
\label{fig:7}
\end{figure}

\begin{figure}[hf!]
\centering
\includegraphics[scale=0.5]{./perturbada/7.jpg}
\caption{cuerda perurbada en T/2 }
\label{fig:8}
\end{figure}






\bibliographystyle{plain}
\bibliography{references}
\end{document}

